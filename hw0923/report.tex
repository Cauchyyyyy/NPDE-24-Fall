%!TEX program = xelatex
\documentclass{article}
\usepackage[UTF8,scheme=plain]{ctex}
\usepackage[a4paper,left=1.25in,right=1.25in,top=1in,bottom=1in]{geometry}

\usepackage{amsmath,amsthm,amsfonts,amssymb}
\usepackage{graphicx}
\usepackage{float}
\usepackage{subcaption}
\usepackage{booktabs,multirow,multicol}
\usepackage{indentfirst}
\usepackage{hyperref}
\usepackage{setspace}
\usepackage{listings}
\usepackage[ruled,noline]{algorithm2e}
\usepackage{bm}
\usepackage{xcolor}


\title{NPDE~第3次实验报告}
\author{Author}
\date{\today}

\begin{document}

\maketitle

\section{Introduction}

对下述偏微分方程初值问题:
$$
\begin{aligned}
    &u_t=u_x,-\infty<x<\infty,t>0\\
    &u(x,0)=sin(2\pi x)\\
    &\text{周期为1}\\
\end{aligned}
$$

该方程的精确解为$u(x,t)=sin(2\pi(x+t))$,对时空区域均匀剖分如下:

时间:$t_n=n\times \Delta t,n=0,1,...,N$,时间步长$\Delta t=1/N$

空间:$x_j=j\times \Delta x,j=0,1,...,J$,空间步长$h=\Delta x=1/J$

令$r=\Delta t/h$

1、取r=0.5,J=80,分别用FTCS,Lax-Friedrich,Lax-Wendroff计算数值解,在t=0.1,0.4,0.8,1时刻求最大误差,
绘制最大误差随时间变化图

2、取r=0.5,t=1.0,分别取J=10,20,40,80,160,用Lax-Wendroff计算数值解
\section{Method}

记近似解$v_j^n=u(x_j,t_n)$,近似方法与对应的离散方程如下:

FTCS:$$v_j^{n+1}=v_j^n+\frac{\Delta t}{2\Delta x}(v_{j+1}^n-v_{j-1}^n)$$

Lax-Friedrich:$$v_j^{n+1}=v_j^n+\frac{\Delta t}{2\Delta x}(v_{j+1}^n-v_{j-1}^n)+\frac{1}{2}(v_{j+1}^n+v_{j-1}^n-2v_j^n)$$

Lax-Wendroff:$$v_j^{n+1}=v_j^n+\frac{\Delta t}{2\Delta x}(v_{j+1}^n-v_{j-1}^n)+\frac{1}{2}(\frac{\Delta t}{\Delta x})^2(v_{j+1}^n+v_{j-1}^n-2v_j^n)$$

\section{Results}

\begin{figure}[H]
    \centering
    \begin{subfigure}{.4\textwidth}
        \includegraphics[width=1\textwidth]{demo1.png}    
    \end{subfigure}
    \begin{subfigure}{.4\textwidth}
        \includegraphics[width=1\textwidth]{demo3.png}        
    \end{subfigure}
    \caption{最大误差随时间变化图}\label{fig:demo1}
\end{figure}

\begin{figure}[H]
    \centering
    \includegraphics[width=.8\textwidth]{demo2.png}
    \caption{不同网格精度的比较(Lax-Wendroff)}\label{fig:demo2}
\end{figure}

\section{Conclusion}

\subsection{问题1}
在给定的时间范围内,最大误差随时间逐渐增加,其中Lax-Wendroff格式的误差最小,其次是FTCS格式、Lax-Friedrich格式。
由理论分析知,三种格式的截断误差分别为$\mathcal{O} (\Delta t+h^2)(FTCS), \mathcal{O} (\Delta t+h)(Lax-Friedrich), \mathcal{O} (\Delta t^2+h^2)(Lax-Wendroff)$,与数值解结果相符。

将时间范围扩大,可以看到FTCS格式是不稳定的,Lax-Friedrich格式、Lax-Wendroff格式是稳定的。
\subsection{问题2}
随着空间步长的减小,Lax-Wendroff格式的数值解逐渐收敛到准确解。
\section{Appendix}
\subsection{Code Discription}
本次作业使用matlab编写,问题1、2分别在main1.m、main2.m文件,运行即可得到结果,实现相关功能所需函数在ndsolve.m文件中
\subsection{Code}
\subsubsection{main1.m}
\lstinputlisting[language=Matlab]{main1.m}
\subsubsection{main2.m}
\lstinputlisting[language=Matlab]{main2.m}
\lstinputlisting[language=Matlab]{ndsolve.m}
\end{document}
