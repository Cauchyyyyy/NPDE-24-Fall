%!TEX program = xelatex
\documentclass{article}
\usepackage[UTF8,scheme=plain]{ctex}
\usepackage[a4paper,left=1.25in,right=1.25in,top=1in,bottom=1in]{geometry}

\usepackage{amsmath,amsthm,amsfonts,amssymb}
\usepackage{graphicx}
\usepackage{float}
\usepackage{subcaption}
\usepackage{booktabs,multirow,multicol}
\usepackage{indentfirst}
\usepackage{hyperref}
\usepackage{setspace}
\usepackage{listings}
\usepackage[ruled,noline]{algorithm2e}
\usepackage{bm}
\usepackage{xcolor}


\title{NPDE~第4次实验报告}
\author{Author}
\date{\today}

\begin{document}

\maketitle

\section{Introduction}

对下述偏微分方程初值问题:
$$
\begin{aligned}
    &u_t=u_x,-\infty<x<\infty,t>0\\
    &u(x,0)=sin(2\pi x)\\
    &\text{周期为1}\\
\end{aligned}
$$

该方程的精确解为$u(x,t)=sin(2\pi(x+t))$,对时空区域均匀剖分如下:

时间:$t_n=n\times \Delta t,n=0,1,...,N$,时间步长$\Delta t=1/N$

空间:$x_j=j\times \Delta x,j=0,1,...,J$,空间步长$h=\Delta x=1/J$

令$r=\Delta t/h$

1、取T=1.0,J=80,分别取r=0.5,1.5,用CTCS格式求数值解
\section{Method}

记近似解$v_j^n=u(x_j,t_n)$,近似方法分别为CTCS,得到离散方程如下:

第一步采用FTFS格式
$$v_j^{n+1}=v_j^n+\frac{\Delta t}{2\Delta x}(v_{j+1}^n-v_{j}^n)$$

之后采用CTCS格式
$$v_j^{n+1}=v_j^{n-1}+\frac{\Delta t}{\Delta x}(v_{j+1}^n-v_{j-1}^n)$$

定解条件:初始条件:$v_j^0=sin(2\pi j\Delta x)$,边界条件:$v_j^n=v_{j+J}^n$


\section{Results}

\begin{figure}[H]
    \centering
    \includegraphics[width=.8\textwidth]{demo1.png}
    \caption{r=0.5}\label{fig:demo1}
\end{figure}

\begin{figure}[H]
    \centering
    \includegraphics[width=.8\textwidth]{demo2.png}
    \caption{r=1.5}\label{fig:demo2}
\end{figure}

\section{Conclusion}

理论分析得,数值解的稳定性条件为$$r\leq 1-\delta ,1>\delta > 0$$

当$r =0.5$时,数值解与准确解误差较小,当$r =1.5$时,数值解不稳定,结果与理论相符。

\section{Appendix}
\subsection{Code Discription}
本次作业使用matlab编写,打开main.m文件,运行即可得到结果,实现相关功能所需函数在ndsolve.m文件中
\subsection{Code}
\lstinputlisting[language=Matlab]{main.m}
\lstinputlisting[language=Matlab]{ndsolve.m}
\end{document}
