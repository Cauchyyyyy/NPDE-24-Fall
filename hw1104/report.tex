%!TEX program = xelatex
\documentclass{article}
\usepackage[UTF8]{ctex}
\usepackage[a4paper,left=1.25in,right=1.25in,top=1in,bottom=1in]{geometry}

\usepackage{amsmath,amsthm,amsfonts,amssymb}
\usepackage{graphicx}
\usepackage{float}
\usepackage{subcaption}
\usepackage{booktabs,multirow,multicol}
\usepackage{indentfirst}
\usepackage{hyperref}
\usepackage{setspace}
\usepackage{listings}
\usepackage[ruled,noline]{algorithm2e}
\usepackage{bm}
\usepackage{xcolor}



\title{NPDE~第5次实验报告}
\author{Author}
\date{\today}

\begin{document}

\maketitle

\section{Introduction}

对以下的偏微分方程初值问题
\begin{equation*}
    \begin{cases}
        u_t+u_x=0, & -\infty < x< \infty,t>0,\\
        u(x,0)=\begin{cases}
            1, & 0.4 \leq x \leq 0.6,\\
            0, & \text{otherwise}\\
            \end{cases}
    \end{cases}
\end{equation*}

1、分别用FTCS、FTBS和Lax-Wendroff格式在t=0.05, 0.2, 0.8, 3.2时刻进行数值求解,
其中$r=\Delta t/\Delta x=0.2, 0.8, \Delta x=0.05$。绘制相同时刻、相同r值下的数值解与精确解图像。

2、分析PDE及其数值格式的色散性和耗散性。
\section{Method}
\subsection{PDE}
$$u_t+u_x=0$$

令 $u(x,t)=e^{i(kt+\omega x)}$,带入PDE得到$k+\omega=0$,方程的谐波解为$u(x,t)=e^{i\omega(x- t)}$。
波速记为$v_0=\frac{-Re k}{\omega}=1$

PDE是无耗散、无色散的。
\subsection{FTCS}
$$v_j^{n+1}=v_j^n-\frac{\Delta t}{2\Delta x}(v_{j+1}^n-v_{j-1}^n)$$

令$v_j^n=e^{i(\omega j\Delta x+k n \Delta t)}$,带入FTCS格式得到$e^{ik\Delta t}=1-irsin(\omega \Delta x)$,
其中$r=\frac{\Delta t}{\Delta x}$。

令$k=a+ib$,则$e^{-b\Delta t}e^{ia\Delta t}=1-irsin(\omega \Delta x)$,
$e^{-b\Delta t}=\sqrt{1+(rsin(\omega \Delta x))^2}>1$,
$a=\frac{1}{\Delta t}arctan(-rsin(\omega \Delta x))$

记$\xi=\omega \Delta x$,当$\xi\longrightarrow \pi, a=0, v=0$,
当$\xi\longrightarrow 0, a=\frac{1}{\Delta t}(-r \xi+\frac{1}{6} \left(2 r^3+r\right) \xi^3+O\left(\xi^4\right))$,
$v=\frac{-a}{\omega}\approx \frac{1}{\omega \Delta t}(r \xi-\frac{1}{6} \left(2 r^3+r\right) \xi^3)=1-\frac{1}{6\omega \Delta t} \left(2 r^3+r\right) \xi^3 \leq v_0$

FTCS格式是逆耗散,有色散的
\subsection{FTBS}
$$v_j^{n+1}=v_j^n-\frac{\Delta t}{\Delta x}(v_j^n-v_{j-1}^n)$$

令$v_j^n=e^{i(\omega j\Delta x+k n \Delta t)}$,带入FTBS格式得到
$$e^{ik\Delta t}=1-r+re^{-i\omega \Delta x}$$

令$k=a+ib$,则$e^{-b\Delta t}e^{ia\Delta t}=1-r+re^{-i\omega \Delta x}$,
$$e^{-b\Delta t}=\sqrt{(1-r+rcos(\xi))^2+(rsin(\xi))^2}$$
$$a=\frac{1}{\Delta t}arctan(\frac{-rsin(\xi)}{1-r+rcos(\xi)})$$

记$\xi=\omega \Delta x$,当$\xi\longrightarrow \pi, a=0, v=0$,

当$\xi\longrightarrow 0, a=\frac{1}{\Delta t}(-r \xi+\frac{1}{6} \left(2 r^3-3 r^2+r\right) \xi^3+O\left(\xi^4\right))$,
$v=\frac{-a}{\omega}\approx \frac{1}{\omega \Delta t}(-r \xi+\frac{1}{6} \left(2 r^3-3 r^2+r\right) \xi^3+O\left(\xi^4\right))
    =1-\frac{1}{6 \omega \Delta t}\left(2 r^3-3 r^2+r\right) \xi^3$

$\frac{1}{2}\leq r \leq 1,v\geq v_0,0\leq r \leq \frac{1}{2},v\leq v_0$

当$r\leq 1$时,FTBS格式是正耗散,有色散的
\subsection{Lax-Wendroff}
$$v_j^{n+1}=v_j^n-\frac{\Delta t}{2\Delta x}(v_{j+1}^n-v_{j-1}^n)+\frac{(\Delta t)^2}{2(\Delta x)^2}(v_{j+1}^n-2v_j^n+v_{j-1}^n)$$

令$v_j^n=e^{i(\omega j\Delta x+k n \Delta t)}$,带入Lax-Wendroff格式得到
$$e^{ik\Delta t}=1-2r^2sin^2\frac{\xi}{2}-irsin\xi$$

令$k=a+ib$,则$e^{-b\Delta t}e^{ia\Delta t}=1-2r^2sin^2\frac{\xi}{2}-irsin\xi$,
$$e^{-b\Delta t}=\sqrt{(1-2r^2sin^2\frac{\xi}{2})^2+(rsin\xi)^2}$$
$$a=\frac{1}{\Delta t}arctan(\frac{-rsin\xi}{1-2r^2sin^2\frac{\xi}{2}})$$

当$\xi\longrightarrow \pi, a=0, v=0$,

当$\xi\longrightarrow 0$, 
$a=\frac{1}{\Delta t}(-r \xi+\frac{1}{6} \left(r-r^3\right) \xi^3+O\left(\xi^4\right))$

$v=\frac{-a}{\omega}=\frac{1}{\omega \Delta t}(r \xi-\frac{1}{6} \left(r-r^3\right) \xi^3+O\left(\xi^4\right))
    \approx 1-\frac{1}{6 \omega \Delta t}(r-r^3)\xi^3\leq v_0$

当$r\leq 1$时,Lax-Wendroff格式是正耗散,有色散的
\section{Results}

数值结果如下
\begin{figure}[htbp]
    \centering
    \begin{subfigure}[b]{0.47\textwidth}
        \centering
        \includegraphics[width=\textwidth]{r2t05.png}
        \caption{t=0.05}
    \end{subfigure}
    \begin{subfigure}[b]{0.47\textwidth}
        \centering
        \includegraphics[width=\textwidth]{r2t20.png}
        \caption{t=0.2}
    \end{subfigure}
    \begin{subfigure}[b]{0.47\textwidth}
        \centering
        \includegraphics[width=\textwidth]{r2t80.png}
        \caption{t=0.8}
    \end{subfigure}
    \begin{subfigure}[b]{0.47\textwidth}
        \centering
        \includegraphics[width=\textwidth]{r2t321.png}
        \caption{t=3.2}
    \end{subfigure}
    \begin{subfigure}[b]{0.47\textwidth}
        \centering
        \includegraphics[width=\textwidth]{r2t322.png}
        \caption{t=3.2}
    \end{subfigure}
    \caption{r=0.2}
    \label{fig:r=0.2}
\end{figure}

\begin{figure}[H]
    \centering
    \begin{subfigure}[b]{0.47\textwidth}
        \centering
        \includegraphics[width=\textwidth]{r8t05.png}
        \caption{t=0.05}
    \end{subfigure}
    \begin{subfigure}[b]{0.47\textwidth}
        \centering
        \includegraphics[width=\textwidth]{r8t20.png}
        \caption{t=0.2}
    \end{subfigure}
    \begin{subfigure}[b]{0.47\textwidth}
        \centering
        \includegraphics[width=\textwidth]{r8t801.png}
        \caption{t=0.8}
    \end{subfigure}
    \begin{subfigure}[b]{0.47\textwidth}
        \centering
        \includegraphics[width=\textwidth]{r8t802.png}
        \caption{t=0.8}
    \end{subfigure}
    \begin{subfigure}[b]{0.47\textwidth}
        \centering
        \includegraphics[width=\textwidth]{r8t32.png}
        \caption{t=3.2}
    \end{subfigure}
    \caption{r=0.8}
    \label{fig:r=0.8}
\end{figure}
\section{Conclusion}

$r=0.2$时,三种数值格式的波速均小于PDE的波速,从图\ref{fig:r=0.2}中可以看出,数值解的波峰的位置落后于准确解。

$r=0.8$时,FTBS格式的波速大于PDE,其他两种仍小于PDE,从图\ref{fig:r=0.8}中可以看出,FTBS格式的波峰位置与准确解位置基本相同。

FTCS格式是逆耗散的,数值解不稳定,所以在$r=0.2,t=0.8$时刻与$r=0.8,t=3.2$时刻给出了除FTCS之外的
图像,由于数值解的耗散性,其最大值是小于1的,$r=0.2,t=3.2$时刻只给出另外两种格式与准确解的图像,FTCS格式此时图像$r=0.2,t=0.8$与基本相同,误差更大
\section{Appendix}
\subsection{Code Discription}

\subsection{Code}
\lstinputlisting[language=python]{main.py}
\end{document}
