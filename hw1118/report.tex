\documentclass{article}
\usepackage[UTF8]{ctex}
\usepackage[a4paper,left=1.25in,right=1.25in,top=1in,bottom=1in]{geometry}

\usepackage{amsmath,amsthm,amsfonts,amssymb}
\usepackage{graphicx}
\usepackage{float}
\usepackage{subcaption}
\usepackage{booktabs,multirow,multicol}
\usepackage{indentfirst}
\usepackage{hyperref}
\usepackage{setspace}
\usepackage{listings}
\usepackage[ruled,noline]{algorithm2e}
\usepackage{bm}
\usepackage{xcolor}


\title{NPDE~第6次实验报告}
\author{Author}
\date{\today}

\begin{document}

\maketitle

\section{Introduction}
对下列PDE进行数值求解:
\begin{equation*}
    \begin{aligned}
        u_t=\nu u_{xx}+F(x,t)&\quad x\in (0,1),t>0\\
        u(x,0)=f(x)&\quad x\in [0,1]\\
        u_x(0,t)=a(t)&\quad t\geq 0\\
        u(1,t)=b(t)&\quad t\geq 0
    \end{aligned}
\end{equation*}

where $\nu = 0.1, f(x) = x(1 - x), a(t) = 10 sin t, b(t) = 4sin 6t$ and $ F(x, t) =
sin 2\pi x sin 4\pi t$.

网格剖分为$0<x_1<...<x_M=1$,分别取$M=10,20,40,\Delta t=0.05,0.01,0.002$,求$t=0.1,0.9,2.0$时的数值结果
\section{Method}
内部点使用FTCS格式,
$$v_{j}^{n+1} = v_{j}^{n} + \frac{\nu \Delta t}{\Delta x^2}(v_{j+1}^{n} - 2v_{j}^{n} + v_{j-1}^{n}) + \Delta t F_{j}^{n}, j=1,...,M-1$$
$$v_M^{n+1} = b(t^{n+1})$$

当$\frac{\nu\Delta t}{\Delta x^2}\leq \frac{1}{2}$时,内部格式是稳定的,题设条件$\nu\Delta t M^2\leq \frac{1}{2}$满足稳定性

边界处理分别采用如下方法:
\subsection{单侧差商离散}
$x_0=0, \Delta x=\frac{1}{M}$,若已求得内部点$v_1^n,...,v_{M-1}^n$,则有
\begin{equation*}
    \begin{aligned}
        \frac{v_1^n-v_0^n}{\Delta x}=a(t^n)\\
        v_0^n=v_1^n-\Delta x a(t^n)
    \end{aligned}
\end{equation*}
\subsection{半网格方法}
$x_0=-\frac{1}{2}\Delta x<0, \Delta x=\frac{1-\frac{1}{2}\Delta x}{M-1}$,若已求得内部点$v_1^n,...,v_{M-1}^n$,则有
\begin{equation*}
    \begin{aligned}
        \frac{v_1^n-v_0^n}{\Delta x}=a(t^n)\\
        v_0^n=v_1^n-\Delta x a(t^n)
    \end{aligned}
\end{equation*}
\section{Results}

得到的数值结果如下:
\begin{figure}[H]
    \centering
    \begin{subfigure}[b]{0.47\textwidth}
        \centering
        \includegraphics[width=\textwidth]{t=0.1,M=10.png}
        \caption{t=0.1,M=10}
    \end{subfigure}
    \begin{subfigure}[b]{0.47\textwidth}
        \centering
        \includegraphics[width=\textwidth]{t=0.1,M=20.png}
        \caption{t=0.1,M=20}
    \end{subfigure}
    \begin{subfigure}[b]{0.47\textwidth}
        \centering
        \includegraphics[width=\textwidth]{t=0.1,M=40.png}
        \caption{t=0.1,M=40}
    \end{subfigure}
    \caption{t=0.1}
\end{figure}
\begin{figure}[H]
    \centering
    \begin{subfigure}[b]{0.47\textwidth}
        \centering
        \includegraphics[width=\textwidth]{t=0.9,M=10.png}
        \caption{t=0.9,M=10}
    \end{subfigure}
    \begin{subfigure}[b]{0.47\textwidth}
        \centering
        \includegraphics[width=\textwidth]{t=0.9,M=20.png}
        \caption{t=0.9,M=20}
    \end{subfigure}
    \begin{subfigure}[b]{0.47\textwidth}
        \centering
        \includegraphics[width=\textwidth]{t=0.9,M=40.png}
        \caption{t=0.9,M=40}
    \end{subfigure}
    \caption{t=0.9}
\end{figure}
\begin{figure}[H]
    \centering
    \begin{subfigure}[b]{0.47\textwidth}
        \centering
        \includegraphics[width=\textwidth]{t=2,M=10.png}
        \caption{t=2.0,M=10}
    \end{subfigure}
    \begin{subfigure}[b]{0.47\textwidth}
        \centering
        \includegraphics[width=\textwidth]{t=2,M=20.png}
        \caption{t=2.0,M=20}
    \end{subfigure}
    \begin{subfigure}[b]{0.47\textwidth}
        \centering
        \includegraphics[width=\textwidth]{t=2,M=40.png}
        \caption{t=2.0,M=40}
    \end{subfigure}
    \caption{t=2.0}
\end{figure}

\section{Conclusion}

总结和思考。

\section{Appendix}
\subsection{Code Discription}

\subsection{Code}
\lstinputlisting[language=python]{main.py}
\end{document}
