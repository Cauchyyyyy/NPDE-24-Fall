%!TEX program = xelatex
\documentclass{article}
\usepackage[UTF8,scheme=plain]{ctex}
\usepackage[a4paper,left=1.25in,right=1.25in,top=1in,bottom=1in]{geometry}
\usepackage{subfigure}
\usepackage{amsmath,amsthm,amsfonts,amssymb}
\usepackage{graphicx}
\usepackage{float}
\usepackage{subcaption}
\usepackage{booktabs,multirow,multicol}
\usepackage{indentfirst}
\usepackage{hyperref}
\usepackage{setspace}
\usepackage{listings}
\usepackage[ruled,noline]{algorithm2e}
\usepackage{bm}
\usepackage{xcolor}

\title{NPDE~第4次实验报告}
\author{Author}
\date{\today}

\begin{document}

\maketitle

\section{Introduction}

对下述偏微分方程初值问题:
$$\left\{
  \begin{aligned}
    u_t&=u_x,\quad -\infty<x<\infty,t>0\\
    u(x,0)&=\sin(2\pi x)\\
    &\text{周期为1}\\
  \end{aligned}
  \right.
  $$

  该方程的精确解为$u(x,t)=\sin(2\pi(x+t))$,对时空区域均匀剖分如下:

  时间:$t_n=n\times \Delta t,n=0,1,...,N$,时间步长$\Delta t$

  空间:$x_j=j\times \Delta x,j=0,1,...,J$,空间步长$h=\Delta x=1/J$

  令$r=\Delta t/h$

  1、取t=1.0,J=80,分别取r=0.5,1.5,用CTCS格式求数值解

  2、取r=0.5,t=1.0,分别取J=10,20,40,80,160,用CTCS格式求数值解

  3、取r=0.5,J=80,分别取t=0.2,0.5,用FTBS格式求数值解
  \section{Method}

  记近似解$v_j^n=u(x_j,t_n)$,近似方法分别为CTCS,得到离散方程如下:

  第一步采用FTFS格式
  $$v_j^{n+1}=v_j^n+\frac{\Delta t}{2\Delta x}(v_{j+1}^n-v_{j}^n)$$

  之后采用CTCS格式
  $$v_j^{n+1}=v_j^{n-1}+\frac{\Delta t}{\Delta x}(v_{j+1}^n-v_{j-1}^n)$$

  定解条件:初始条件:$v_j^0=\sin(2\pi j\Delta x)$,边界条件:$v_j^n=v_{j+J}^n$

  \section{Results}
  CTCS格式结果如下:
  \begin{figure}[H]
    \centering
    \subfigure[r=0.5]{
    \includegraphics[width=0.45\textwidth]{CTCS, r = 0.5, t=1.0, J=80.pdf}}
    \subfigure[r=1.5]{
    \includegraphics[width=0.45\textwidth]{CTCS, r = 1.5, t=1.0, J=80.pdf}}
    \caption{CTCS, t=1.0, J=80}
  \end{figure}

  \begin{figure}[H]
    \centering
    \subfigure[J=10]{
    \includegraphics[width=0.45\textwidth]{CTCS, r = 0.5, t=1.0, J=10.pdf}}
    \subfigure[J=20]{
    \includegraphics[width=0.45\textwidth]{CTCS, r = 0.5, t=1.0, J=20.pdf}}
    \subfigure[J=40]{
    \includegraphics[width=0.45\textwidth]{CTCS, r = 0.5, t=1.0, J=40.pdf}}
    \subfigure[J=80]{
    \includegraphics[width=0.45\textwidth]{CTCS, r = 0.5, t=1.0, J=80.pdf}}
    \subfigure[J=160]{
    \includegraphics[width=0.45\textwidth]{CTCS, r = 0.5, t=1.0, J=160.pdf}}
    \caption{CTCS, r=0.5, t=1.0}
  \end{figure}
  FTBS格式结果如下:
  \begin{figure}[H]
    \centering
    \subfigure[t=0.2]{
    \includegraphics[width=0.45\textwidth]{FTBS, r = 0.5, t=0.2, J=80.pdf}}
    \subfigure[t=0.5]{
    \includegraphics[width=0.45\textwidth]{FTBS, r = 0.5, t=0.5, J=80.pdf}}
    \caption{FTBS, r=0.5, J=80}
  \end{figure}
  \section{Conclusion}

  理论分析得,CTCS格式数值解的稳定性条件为$$r\leq 1-\delta ,1>\delta > 0$$

  当$r =0.5$时,数值解与准确解误差较小,当$r =1.5$时,数值解不稳定,FTBS格式是不稳定的,结果与理论相符。

  \section{Appendix}
  \subsection{Code Discription}
  本次作业使用python编写,打开main.py文件,运行即可得到结果
  \subsection{Code}
  \lstinputlisting[language=Python]{main.py}
  \end{document}
